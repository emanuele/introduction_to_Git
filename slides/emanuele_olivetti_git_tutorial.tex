\documentclass{beamer}
\usepackage{xspace}

\mode<presentation>
%\mode<handout>
{
  \usetheme{Luebeck}
  \usecolortheme{beaver}
  % \usetheme{Warsaw}
  % \usecolortheme{default}
  % or ...

  \setbeamercovered{transparent}
  % or whatever (possibly just delete it)
}

% Enable Bold Typewriter font (\texttt{\textbf{}})
\DeclareFontShape{OT1}{cmtt}{bx}{n}{<5><6><7><8><9><10><10.95><12><14.4><17.28><20.74><24.88>cmttb10}{}

\usepackage[english]{babel}
\usepackage[latin1]{inputenc}
\usepackage{times}
\usepackage[T1]{fontenc}
\usepackage{graphicx}
\newcommand{\Ivec}[1]{\mbox{\boldmath $#1$}}
\usepackage[normalem]{ulem}
% \usepackage{tipa}
\newcommand{\git}{\texttt{\textbf{git}}\xspace}
\usepackage{listings}
\usepackage{xcolor}

\definecolor{grey}{rgb}{0.5,0.5,0.5}

\setbeamertemplate{footline}[page number]

% Or whatever. Note that the encoding and the font should match. If T1
% does not look nice, try deleting the line with the fontenc.

\title[A Practical Introduction to \git]{A Practical Introduction to \git}

\author[Emanuele Olivetti]{Emanuele Olivetti}

\institute[FBK/CIMeC]
{
  NeuroInformatics Laboratory (NILab)\\
  Bruno Kessler Foundation (FBK), Trento, Italy\\
  Center for Mind and Brain Sciences (CIMeC),
  University of Trento, Italy\\
  \url{http://nilab.fbk.eu} \\
  \url{olivetti@fbk.eu}
}


\date[Git Tutorial] % (optional, should be abbreviation of conference name)

% If you have a file called "university-logo-filename.xxx", where xxx
% is a graphic format that can be processed by latex or pdflatex,
% resp., then you can add a logo as follows:

%% \pgfdeclareimage[height=1.0cm]{fbk-logo}{figs/Logo-fbk}
%% \logo{\pgfuseimage{fbk-logo}}
%% \pgfdeclareimage[height=1.0cm]{unitn-alfa-logo}{figs/Logo-unitn-alfa}
%% \logo{\pgfuseimage{unitn-alfa-logo}}


% Delete this, if you do not want the table of contents to pop up at
% the beginning of each subsection:


\AtBeginSection[]
{
  \begin{frame}<beamer>{Outline}
    \tableofcontents[currentsection,currentsubsection]
  \end{frame}
}


% If you wish to uncover everything in a step-wise fashion, uncomment
% the following command: 

% \beamerdefaultoverlayspecification{<+->}


\begin{document}

\lstset{language=bash}

\begin{frame}
  \titlepage
\end{frame}

\begin{frame}{Outline}
  \begin{itemize}
  \item Version Control: \git.
  \item Scenario 1: \textbf{single} developer, \textbf{local}
    repository.
    \begin{itemize}
    \item Demo \textbf{single+local}
    \end{itemize}
  \item Scenario 2: \textbf{Team} of developers, \textbf{central
      remote} repository. Minimalistic.
    \begin{itemize}
    \item Demo \textbf{multi+remote}
    \end{itemize}
  \item Branching.
  \item Scenario 3: Contributing to a Software Project hosted on
    \textbf{GitHub}.
  \item Extras: how to set up centralised repository, and more.
  \end{itemize}
\end{frame}

\begin{frame}{Version Control: Naming \& Meaning}
  \begin{block}{Wikipedia}
    ``\emph{Revision control}, also known as \emph{version control},
    \emph{source control} or \emph{software configuration management}
    (SCM), is the \alert{management of changes to documents, programs,
      and other information stored as computer files}.''
  \end{block}
  Popular Acronyms:
  \begin{itemize}
  \item VC
  \item SCM
  \end{itemize}
  Misnomer:
  \begin{itemize}
  \item \sout{Versioning}
  \end{itemize}
  \textbf{Q:} have you ever used VC? (raise your hand = YES)
\end{frame}

\begin{frame}{Version Control: Local, Centralized, Distributed}
  \includegraphics[width=1.05\linewidth]{figs/vc_paradigms}
\\
\small
From \emph{Pro Git}, S.Chacon 2009, CC 3.0 license.
\end{frame}

% \begin{frame}{Distributed Version Control}
%   \begin{block}{Wikipedia: Distributed revision control (or
%       Distributed Version Control (Systems) (DVCS), or Decentralized
%       Version Control)}
%     \emph{``A fairly recent innovation in software revision
%       control. [...] The line between distributed and centralized
%       systems is blurring in some regards, especially since DVCSs can
%       be used in a centralized mode.''}
%   \end{block}
%   \begin{itemize}
%   \item \color{white!30!black}There may be \textbf{many central} repositories.
%   \item \color{white!50!black}Codes from disparate repositories are merged.
%   \item \color{white!70!black}Lieutenants dynamically decide which branches to merge.
%   \item \color{white!80!black}Network is not involved in most operations.
%   \item \color{white!90!black}\emph{sync} operations are available for
%     committing or receiving changes with remote repositories.
%   \end{itemize}
% \end{frame}

% \begin{frame}{Why \git?}
% From: \emph{2010 Python Bootcamp}, Day 3 (Peter Williams).
%   \begin{center}
%     \textbf{Why We Like Git}
%   \end{center}
%   \begin{itemize}
%   \item \color{white!20!black}{Fundamental reason: extremely
%       well-engineered, underlying theory is solid. (Turns out Linus
%       knows what he's doing.)}
%   \item \color{white!30!black}{Rock-solid reliability}
%   \item \color{white!40!black}{Very, very fast}
%   \item \color{white!50!black}{Open-source and Free software}
%   \item \color{white!60!black}{Ergonomic}
%   \item \color{white!70!black}{Extremely powerful suite of tools}
%   \item \color{white!80!black}{Decentralized code-sharing model}
%   \item \color{white!90!black}{Active, committed developer community}
%   \item \color{white!95!black}{Secure}
%   \end{itemize}
% \end{frame}

\begin{frame}{Survey: \git}
  \begin{itemize}\setlength{\itemsep}{+3mm}
  \item Q1: Have you heard about \git?
  \item Q2: Do you use \git?
  \item Q3: Why the ``\git'' name? (from \git FAQ)
    \begin{enumerate}\setlength{\itemsep}{+2mm}
    \item Random three-letter combination that is pronounceable.
    \item Acronym (global information tracker).
    \item Irony.
    \end{enumerate}
  \end{itemize}
\end{frame}

\begin{frame}{\git? Why ``\git''?}
  \begin{columns}
    \begin{column}{5cm}
      \textbf{Linus Torvalds}: 
      % ``\emph{I'm an egotistical [sic] bastard, and
        \emph{``I name all my projects after myself. First Linux, now
        \git.}''
      \begin{figure}
        \centering
        \includegraphics[width=4cm]{figs/Linus_Torvalds}
      \end{figure}
    \end{column}
    \begin{column}{7cm}
      \begin{figure}
        \url{http://www.merriam-webster.com/dictionary/git}
        \centering
        \includegraphics[width=7cm]{figs/git-merriam}
      \end{figure}
    \end{column}
    \end{columns}
\end{frame}


\begin{frame}[containsverbatim]{\git}
  \begin{center}
  \item \texttt{\textbf{git}}
  \end{center}
\begin{verbatim}
usage: git [OPTIONS] COMMAND [ARGS]

The most commonly used git commands are:
   add        Add file contents to the index
   commit     Record changes to the repository
   diff       Show changes between commits, commit and working tree, etc
   ...
\end{verbatim}
  \begin{center}
    \texttt{\textbf{git help <command>}}
  \end{center}
  \begin{center}
    \texttt{\textbf{git status}}
  \end{center}
\end{frame}

\begin{frame}{\git}
  Introduce yourself to \git:
  \begin{center}
    \small
    \texttt{\textbf{git config -{}-global user.name "Emanuele Olivetti"}}
  \end{center}
  \vspace{0.2cm}
  \begin{center}
    \small
    \texttt{\textbf{git config -{}-global user.email "olivetti@fbk.eu"}}
  \end{center}
\end{frame}

\begin{frame}{\git. Single developer + local repository.}
  \begin{figure}
    \centering
    Scenario 1: single developer + local repository.
  \end{figure}
\end{frame}

\begin{frame}{Single+Local \git. Motivations.}
  \begin{itemize}\setlength{\itemsep}{+3mm}
  \item \textbf{Q:} do you use VC for local repo?
  \item Why VC for single developer + local repository?
    \begin{itemize}\setlength{\itemsep}{+2mm}
    \item First step towards a shared project.
    \item Backup.
    \item To keep the memory of your work.
    \end{itemize}
  \end{itemize}
\end{frame}

\begin{frame}{Single+Local \git. Init.}
  \begin{center}
    \texttt{\textbf{git init}}
  \end{center}
  \begin{itemize}
  \item Creates an empty \git repository.
  \item Creates the git directory: \texttt{.git/}
  \end{itemize}
  \begin{figure}
    \centering
    \includegraphics[width=9cm]{figs/local-init}
  \end{figure}
  Note: it is \textbf{safe}. It does not change your pre-existing
  files.
\end{frame}

\begin{frame}{Single+Local \git. The tracking process.}
  \begin{center}
    \texttt{\textbf{git add <filename>}}
  \end{center}
  \begin{figure}
    \centering
    \includegraphics[width=6.5cm]{figs/staging}
  \end{figure}
  \begin{center}
    \texttt{\textbf{git commit -m "Let us begin."}}
  \end{center}

  \begin{block}{Wikipedia}
    ``A \emph{staging area} is a location where organisms, people,
    vehicles, equipment or material are assembled before use''.
  \end{block}
\end{frame}

\begin{frame}{Single+Local \git. Add.}
  \begin{center}
    \texttt{\textbf{git add file1 [file2 ...]}}
  \end{center}
  \begin{itemize}
  \item Adds new files for next commit.
  \item Adds content from working dir to the staging area (index) for
    next commit.
  \item DOES NOT add info on file permissions other than \emph{exec/noexec}
    (755 / 644).
  \item DOES not add directories \emph{per se}.
  \end{itemize}
  \begin{figure}
    \centering
    \includegraphics[width=9cm]{figs/local-add}
  \end{figure}
\end{frame}

\begin{frame}{Single+Local \git. Commit.}
  \begin{center}
    \texttt{\textbf{git commit [-m "Commit message."]}}
  \end{center}
  Records changes from the staging area to master.
  \begin{figure}
    \centering
    \includegraphics[width=9cm]{figs/local-commit}
  \end{figure}
\end{frame}

\begin{frame}{Single+Local \git. Commit.}
  \begin{center}
    \texttt{\textbf{git commit file1 file2}}
  \end{center}
  Records all changes of \texttt{\textbf{file1}},
  \texttt{\textbf{file2}} from working dir and staging area to master.
  \begin{figure}
    \centering
    \includegraphics[width=9cm]{figs/local-commit-filename}
  \end{figure}
  \begin{center}
    \texttt{\textbf{git commit -a}}
  \end{center}
  Records all changes in working dir and staging area. \emph{Be Careful!}
\end{frame}

\begin{frame}[containsverbatim]{Single+Local \git. Commit names. \alert{OPTIONAL}}
  \begin{itemize}
  \item Every \emph{commit} is a \git-object.
  \item The history of a project is a graph of objects referenced by a
    40-digit \git-name: $SHA1(object)$.
  \item $SHA1(object)$ = 160-bit Secure Hash Algorithm.
  \item Examples:
\begin{verbatim}
$ git commit README -m "Added README."
\end{verbatim}
\texttt{[master \alert{dbb4929}] Added README.}
\begin{verbatim}
1 files changed, 1 insertions(+), ...
\end{verbatim}
or
\begin{verbatim}
$ git log
\end{verbatim}
\texttt{commit \alert{dbb49293790b84f0bdcd74fd9fa5cab0...}}
\begin{verbatim}
Author: Emanuele Olivetti <olivetti@fbk.eu>
Date:   Wed Sep 15 00:08:46 2010 +0200
...
\end{verbatim}
  \end{itemize}
\end{frame}
% $

\begin{frame}{Single+Local \git. Diff.}
  \begin{center}
    \texttt{\textbf{git diff}}
  \end{center}
  Shows what changes between \emph{working directory} and
  \emph{staging area} (\emph{index}).
  \begin{figure}
    \centering
    \includegraphics[width=9cm]{figs/local-diff}
  \end{figure}
\end{frame}

\begin{frame}{Single+Local \git. Diff. \alert{OPTIONAL}}
  \begin{block}{\textbf{Q:} ``\texttt{\textbf{git add}}'' then
      ``\texttt{\textbf{git diff}}''. What output?}
    \texttt{\textbf{git diff -{}-staged}} shows differences between
    index and last commit (\alert{HEAD}).
  \end{block}
  \begin{figure}
    \centering
    \includegraphics[width=9cm]{figs/local-diff-cached}
  \end{figure}
\end{frame}


\begin{frame}{Single+Local \git. Logs.}
  \begin{center}
    \texttt{\textbf{git log}}
  \end{center}
  Shows details of the commits.
  \begin{figure}
    \centering
    \includegraphics[width=9cm]{figs/local-log}
  \end{figure}
\end{frame}

\begin{frame}{Single+Local \git. Logs.}
  \begin{center}
    \texttt{\textbf{gitk}}
  \end{center}
  GUI to browse the \git repository.
  \begin{figure}
    \centering
    \includegraphics[width=9.4cm]{figs/gitk_cropped}
  \end{figure}
\end{frame}


\begin{frame}{Single+Local \git. ``How to clean this mess??'' \alert{OPT.}}
  \begin{center}
    \texttt{\textbf{git checkout <filename>}}
  \end{center}
  Get rid of what changed in \texttt{\textbf{<filename>}} (between
  working dir and staging area).
  \begin{figure}
    \centering
    \includegraphics[width=9cm]{figs/local-checkout-file}
  \end{figure}
\end{frame}


\begin{frame}{Single+Local \git. Time travelling. \alert{OPTIONAL}}
  Back to the past when you did commit \texttt{\alert{dbb49293790b84...}}
  \begin{center}
    \texttt{\textbf{git checkout \alert{dbb4929}}}
  \end{center}
  ...and now, \emph{back to the present!}
  \begin{center}
    \texttt{\textbf{git checkout \alert{master}}}
  \end{center}
\end{frame}


\begin{frame}{Single+Local \git. ``How to clean this mess??''. \alert{OPT.}}
  First read \emph{carefully} \texttt{\textbf{git status}}. If you panic:
  \begin{center}
    \texttt{\textbf{git reset -{}-hard HEAD}}
  \end{center}
  Restore all files as in the last commit.
  \begin{figure}
    \centering
    \includegraphics[width=9cm]{figs/local-reset-hard}
  \end{figure}
Warning: \alert{\texttt{reset} can destroy history!}
\end{frame}

\begin{frame}{Single+Local \git. (Re)move. \alert{OPTIONAL}}
  \alert{Warning}: whenever you want to \emph{remove}, \emph{move} or
  \emph{rename} a tracked file use \git:
  \begin{center}
    \texttt{\textbf{git rm <filename>}}
  \end{center}
  \begin{center}
    \texttt{\textbf{git mv <oldname> <newname>}}
  \end{center}
Remember to \texttt{\textbf{commit}} these changes!
  \begin{center}
    \texttt{\textbf{git commit -m "File (re)moved."}}
  \end{center}
\end{frame}


\begin{frame}{Single+Local \git. Demo.}
  \begin{center}
    Demo: \texttt{\textbf{demo\_git\_single\_local.txt}}
  \end{center}
\end{frame}


\begin{frame}{multi+remote/shared \git.}
  \begin{figure}
    \centering
    Scenario 2: multiple developers + remote central repository.
  \end{figure}
\end{frame}

\begin{frame}{multi+remote/shared \git.}
  \begin{figure}
    \centering
    \includegraphics[width=10cm]{figs/workflow-a}
  \end{figure}
\end{frame}

\begin{frame}{multi+remote/shared \git.}
  \begin{figure}
    \centering
    \includegraphics[width=11cm]{figs/local-remote}
  \end{figure}  
\end{frame}

\begin{frame}{multi+remote/shared \git.}
  \begin{center}
    \texttt{\textbf{git clone <URL>}}
  \end{center}
  \uncover<2->{Creates \alert{\emph{two}} local copies of the
    \alert{whole} remote repository.}
  \begin{figure}
    \begin{center}
      \includegraphics<1>[width=9cm]{figs/local-remote-before-clone}
      \includegraphics<2>[width=9cm]{figs/local-remote-clone2}
    \end{center}
  \end{figure}
  \uncover<2->{Available transport protocols:
  \begin{itemize}
  \item \texttt{\textbf{ssh://}}, \texttt{\textbf{git://}},
    \texttt{\textbf{http://}}, \texttt{\textbf{https://}},
    \texttt{\textbf{file://}}
  \end{itemize}
  Ex.: \small{\texttt{\textbf{git clone https://github.com/ASPP/pelita.git}}}
  \begin{center}
    \texttt{\textbf{git remote -v}}
  \end{center}
  shows \textbf{name} and \texttt{\textbf{URL}} of the remote repository.}
\end{frame}

\begin{frame}{multi+remote/shared \git. Fetch.}
  \begin{center}
    \texttt{\textbf{git fetch}}
  \end{center}
  \begin{itemize}
  \item Downloads updates from remote master to local remote master.
  \item The local master, staging area and working directory do not
    change.
  \end{itemize}
  \begin{figure}
    \centering
    \includegraphics[width=9.5cm]{figs/local-remote-fetch}
  \end{figure}
  \begin{block}{\small{\textbf{Q:} Why \textbf{origin}?}}
      \small{\textbf{A:} Just a label for \textbf{Remote}. Choose the one you like.}
  \end{block}
\end{frame}


\begin{frame}{multi+remote/shared \git. Merge.}
  \begin{center}
    \texttt{\textbf{git merge}}
  \end{center}
  \begin{itemize}
  \item Joins development histories together.
  \item \alert{Warning}: can generate \emph{conflicts}!
  \item \textbf{Note}: it merges only when all changes are committed.
  \end{itemize}
  \begin{figure}
    \centering
    \includegraphics[width=9.5cm]{figs/local-remote-merge}
  \end{figure}
  \begin{center}
    \texttt{\textbf{git fetch}} + \texttt{\textbf{git merge}} = \texttt{\textbf{git pull}}
  \end{center}
\end{frame}

\begin{frame}[containsverbatim]{multi+remote/shared \git. Conflicts.}
  \begin{center}
    \alert{Conflict!}
  \end{center}
\begin{verbatim}
  ...
  <<<<<<< yours:sample.txt
  Conflict resolution is hard;
  let's go shopping.
  =======
  Git makes conflict resolution easy.
  >>>>>>> theirs:sample.txt
  ...
\end{verbatim}
\end{frame}


\begin{frame}{multi+remote/shared \git. Conflicts.}
How to resolve conflicts.
\begin{enumerate}
\item See where conflicts are:
  \begin{center}
  \texttt{\textbf{git diff}}
  \end{center}
\item Edit conflicting lines.
\item Add changes to the staging area:
  \begin{center}
    \texttt{\textbf{git add file1 [...]}}
  \end{center}
\item Commit changes:
  \begin{center}
    \texttt{\textbf{git commit -m "Conflicts solved."}}
  \end{center}
\end{enumerate}
\end{frame}

\begin{frame}{multi+remote/shared \git.}
  \begin{center}
    \texttt{\textbf{git push}}
  \end{center}
  \begin{itemize}
  \item Updates \emph{remote masters} (both Local and Remote).
  \item Requires \texttt{\textbf{fetch}}+\texttt{\textbf{merge}} first.
  \end{itemize}
  \begin{figure}
    \centering
    \includegraphics[width=9.5cm]{figs/local-remote-push}
  \end{figure}
\end{frame}

\begin{frame}{multi+remote/shared \git.}
  \begin{center}
    Demo: \texttt{\textbf{demo\_git\_multi\_remote.txt}}.
  \end{center}
  Other related files:
  \begin{itemize}
  \item \texttt{\textbf{create\_remote\_repo\_sn.sh}}
  \item \texttt{\textbf{collaborator1.sh}}
  \item \texttt{\textbf{collaborator2.sh}}
  \item \texttt{\textbf{collaborator2.sh}}
  \end{itemize}
\end{frame}


\begin{frame}{Branching.}
  \begin{figure}
    \centering
    basic branching
  \end{figure}
\end{frame}

\begin{frame}{Branching.}
  \begin{columns}
    \begin{column}{0.6\linewidth}
      \begin{center}
        \includegraphics[width=6cm]{figs/18333fig0310-tn}
      \end{center}
    \end{column}
    \begin{column}{0.4\linewidth}
      \begin{center}
        \texttt{\textbf{git commit (C0)}}\\
        \texttt{\textbf{git commit (C1)}}\\
        \texttt{\textbf{git commit (C2)}}
      \end{center}
    \end{column}
  \end{columns}
  \vspace{1em}
  \begin{center}
    \includegraphics[width=9cm]{figs/local-remote-clone}
  \end{center}
\end{frame}

\begin{frame}{Branching.}
  \begin{columns}
    \begin{column}{0.6\linewidth}
      \begin{center}
        \includegraphics[width=6cm]{figs/18333fig0311-tn}
      \end{center}
    \end{column}
    \begin{column}{0.45\linewidth}
        \uncover<1->\texttt{\textbf{git branch iss53}}\\
        \uncover<2->\texttt{\textbf{git checkout iss53}}\\
        \uncover<3->\texttt{\textbf{git checkout master}}
    \end{column}
  \end{columns}
  \begin{center}
    \includegraphics<1>[width=9cm]{figs/git-branch}
    \includegraphics<2>[width=9cm]{figs/git-checkout}
    \includegraphics<3>[width=9cm]{figs/git-branch}
  \end{center}
\end{frame}

\begin{frame}{Branching.}
  \begin{columns}
    \begin{column}{0.6\linewidth}
      \begin{center}
        \includegraphics[width=6cm]{figs/18333fig0312-tn}
      \end{center}
    \end{column}
    \begin{column}{0.4\linewidth}
      \begin{center}
        \texttt{\textbf{git checkout iss53}}\\
        \texttt{\textbf{git commit (C3)}}
      \end{center}
    \end{column}
  \end{columns}
  \begin{center}
    \includegraphics[width=9cm]{figs/git-checkout}
  \end{center}
\end{frame}

\begin{frame}{Branching.}
  \begin{columns}
    \begin{column}{0.6\linewidth}
      \begin{center}
        \includegraphics<1>[width=5.5cm]{figs/18333fig0313b-tn}
        \includegraphics<2>[width=5.5cm]{figs/18333fig0313-tn}
      \end{center}
    \end{column}
    \begin{column}{0.45\linewidth}
      \begin{center}
        \uncover<1->\texttt{\textbf{git branch hotfix}}\\
        \uncover<2->\texttt{\textbf{git checkout hotfix}}\\
        \uncover<2->\texttt{\textbf{git commit (C4)}}\\
      \end{center}
    \end{column}
  \end{columns}
  \begin{center}
    \includegraphics<1>[width=9cm]{figs/git-branch2}
    \includegraphics<2>[width=9cm]{figs/git-checkout2}
  \end{center}
\end{frame}

\begin{frame}{Branching.}
  \begin{columns}
    \begin{column}{0.6\linewidth}
      \begin{center}
        \includegraphics[width=6.5cm]{figs/18333fig0314c-tn}
      \end{center}
    \end{column}
    \begin{column}{0.45\linewidth}
      \begin{center}
        \texttt{\textbf{git checkout master}}\\
        \texttt{\textbf{git merge hotfix}}\\
      \end{center}
    \end{column}
  \end{columns}
  \begin{center}
    \includegraphics[width=9cm]{figs/git-branch2}
  \end{center}
\end{frame}

\begin{frame}{Branching.}
  \begin{columns}
    \begin{column}{0.6\linewidth}
      \begin{center}
        \includegraphics[width=6cm]{figs/18333fig0315-tn}
      \end{center}
    \end{column}
    \begin{column}{0.4\linewidth}
      \begin{center}
        \texttt{\textbf{git checkout iss53}}\\
        \texttt{\textbf{git commit (C5)}}
      \end{center}
    \end{column}
  \end{columns}
  \begin{center}
    \includegraphics[width=9cm]{figs/git-checkout}
  \end{center}
\end{frame}

\begin{frame}{Branching.}
  \begin{columns}
    \begin{column}{0.6\linewidth}
      \begin{center}
        \includegraphics[width=6cm]{figs/18333fig0317-tn}
      \end{center}
    \end{column}
    \begin{column}{0.45\linewidth}
      \begin{center}
        \texttt{\textbf{git checkout master}}\\
        \texttt{\textbf{git merge iss53}}
      \end{center}
    \end{column}
  \end{columns}
  \begin{center}
    \includegraphics[width=9cm]{figs/git-branch2}
  \end{center}
\end{frame}


\begin{frame}{Contributing through GitHub}
  \begin{figure}
    \centering
    Scenario 3: contributing to a software project hosted on GitHub.
  \end{figure}
\end{frame}

\begin{frame}{Contributing through GitHub}
  \textbf{Q:} Have you ever heard of GitHub?
  \includegraphics[width=11cm]{figs/github}
  \begin{block}{What is GitHub?}
    \begin{itemize}
    \item Wikipedia: \emph{``GitHub is a web-based hosting service for
        software development projects that use \git''}.
      % remember to mention that now it is not just for git, but also
      % svn and mercurial (thanks Zbyszek).
    \item 5 millions repositories (Jan 2013).
    \item Commercial...
    \item ...but friendly to Free / Open Source software projects.
    \end{itemize}
  \end{block}
\end{frame}

\begin{frame}{Contributing through GitHub}
  \begin{block}{Assumptions}
    \begin{itemize}
    \item You use a software and feel ready to contribute to it.
    \item The software project is hosted on \url{http://github.com}
    \end{itemize}
  \end{block}
  \begin{block}{Intuitive Idea}
    \begin{itemize}
    \item You \textbf{do not push} your changes to the main
      repository.
    \item Instead you create a public copy (\textbf{fork}) of the main
      repository...
    \item ...and then push your changes to that.
    \item Then you ask the owners of the main repository if they like
      your changes and want to merge them (\textbf{pull request}).
    \end{itemize}
  \end{block}
\end{frame}

\begin{frame}{Contributing through GitHub. Not for everyone ;-)}
\includegraphics[width=8cm]{figs/linus-on-pull-requests}
\begin{center}
  \url{https://github.com/torvalds/linux/pull/17}
\end{center}
\end{frame}

\begin{frame}{Contributing through GitHub: Recipe I}
  \begin{enumerate}
  \item \textbf{Register} on \url{http://github.com}
  \item \textbf{Visit} the GitHub page of the software project and
    \textbf{Fork} it:
    \hbox{\hspace{-1cm}\includegraphics[width=11cm]{figs/github-fork}}
  \item \textbf{Clone your copy} of the project on your computer.
    \begin{center}
      \begin{footnotesize}
        \texttt{\textbf{git clone git@github.com:<login>/<project>.git}}
      \end{footnotesize}
    \end{center}
  \item Create a \textbf{branch} to host your improvements.
    \begin{itemize}
    \item \texttt{\textbf{git branch <new-feature>}} \includegraphics[height=0.55cm]{figs/github-branch}
    \item \texttt{\textbf{git checkout <new-feature>}}
    \end{itemize}
  \end{enumerate}
\end{frame}

\begin{frame}{Contributing through GitHub: Recipe II}
  \begin{enumerate}
    \setcounter{enumi}{4}
  \item Add your improvements.
    \begin{itemize}
    \item \texttt{\textbf{git add <new-file>}}
    \item \texttt{\textbf{git commit -m ...}}
    \end{itemize}    
  \item \textbf{Push} your improvements.
    \begin{center}
      \texttt{\textbf{git push origin <new-feature>}}
    \end{center}
  \item Send a \textbf{pull request}.
    \includegraphics[height=0.55cm]{figs/github-compare-and-pull-request}
    \begin{center}
      \hbox{\hspace{-0cm}\includegraphics[width=11cm]{figs/github-pull-request}}
    \end{center}
  \end{enumerate}
\end{frame}

\begin{frame}{Contributing through GitHub}
  \begin{center}
    Detailed Explanation
\end{center}
\end{frame}

\begin{frame}{Contributing through GitHub: Detailed Explanation}
  \includegraphics<1>[width=11.5cm]{figs/pull-request-0}
  \includegraphics<2>[width=11.5cm]{figs/pull-request3-1}
  \includegraphics<3>[width=11.5cm]{figs/pull-request3-2}
  \includegraphics<4>[width=11.5cm]{figs/pull-request3-3}
  \includegraphics<5>[width=11.5cm]{figs/pull-request3-4}
  \includegraphics<6>[width=11.5cm]{figs/pull-request3-5}
  \includegraphics<7>[width=11.5cm]{figs/pull-request-4}
  \includegraphics<8>[width=11.5cm]{figs/pull-request-5}
  \begin{center}
    \only<1>{There is a software project hosted on remote GitHub
      repository (\textbf{upstream}). You want to improve it.}
    \only<2>{So you \textbf{fork} it by creating a (remote) copy of
      it:\\
        \textcolor{grey}{\texttt{\textbf{git clone -{}-bare
            <UPSTREAM\_URL>}}}}
    \only<3>{Now you clone your copy on your local computer:\\
      \texttt{\textbf{git clone <ORIGIN\_URL>}}}
    \only<4>{\texttt{\textbf{git remote add upstream
          <UPSTREAM\_URL>}}\\
    \texttt{\textbf{git fetch upstream}}}
    \only<5>{\texttt{\textbf{git branch new-feature upstream/master}}\\
      \texttt{\textbf{git checkout new-feature}}}
    \only<6>{\texttt{\textbf{git add ... \\git commit ...}}}
    \only<7>{publish your new feature:\\
      \texttt{\textbf{git push origin new-feature}}} 
    \only<8>{Notify the owners of the main repository about
      \texttt{\textbf{new-feature}}\\
      they: \texttt{\textbf{git fetch}} + (eventually)
      \texttt{\textbf{git merge}}}
  \end{center}
\end{frame}

% \begin{frame}{Contributing through GitHub. Demo.}
%   \begin{center}
%     Demo: Live.
%   \end{center}
% \end{frame}

% \begin{frame}{Extras. \alert{OPTIONAL}}
%   \begin{itemize}
%   \item Local branching + demo.
%   \item Setting up a remote shared repository + demo.
%   \end{itemize}
% \end{frame}

\begin{frame}{Setting up a remote+shared repository. \alert{OPTIONAL}}
  \begin{center}
    GOAL: I want to share my local repository so others can \texttt{\textbf{push}}.
  \end{center}
  ``Why can't I just \emph{extend permissions} in my \textbf{local} repo?''
  \begin{itemize}
  \item Yes you can...
  \item ...but your colleagues will not push (\alert{read-only}).
  \end{itemize}
  \begin{center}
    To have it \alert{read-write}: set up a \textbf{remote}
    \emph{shared} repository.
  \end{center}
  \begin{figure}
    \centering
    \includegraphics[width=9cm]{figs/workflow-a}
  \end{figure}
\end{frame}

\begin{frame}{Setting up a remote+shared repository. \alert{OPTIONAL}}
  You have a local repository and want to share it
  (\texttt{\textbf{ssh}}) from a remote server on which your
  colleagues already have access.
  \begin{block}{On \emph{remote} server create
      \textbf{bare}+\textbf{shared} repository:}
    \begin{itemize}
    \item \texttt{\textbf{mkdir newproject}}
    \item set up proper \emph{group} permissions: \texttt{\textbf{chmod g+rws newproject}}
    \item \texttt{\textbf{cd newproject}}
    \item \texttt{\textbf{git \alert{-{}-bare} init \alert{-{}-shared=group}}}
    \end{itemize}
  \end{block}
  \begin{block}{On \emph{local} machine push your repository to remote:}
    \begin{itemize}
    \item \texttt{\textbf{git remote add origin }}
          \begin{center}
            \texttt{\textbf{ssh://remote.com/path/newproject}}
      \end{center}
    \item \texttt{\textbf{git push \alert{-u} origin master}}
    \end{itemize}
  \end{block}
  Everybody clones the shared repository:
  \begin{center}
    \texttt{\textbf{git clone ssh://remote.com/path/newproject}}
  \end{center}
\end{frame}


\begin{frame}{Setting up a remote+shared repository. \alert{OPTIONAL}}
  \begin{center}
    Demo: \texttt{\textbf{demo\_git\_setup\_remote.txt}}.
  \end{center}
\end{frame}

\begin{frame}{remote+shared repository is not secure. \alert{OPTIONAL}}
  \begin{center}
    An user can intentionally (\texttt{ssh + rm}) or less
    intentionally \texttt{git push --force} corrupt or destroy data on
    the remote repositories.

    For this reason there are specialized tools to enforce fine
    grained access, see for example:
    \url{http://gitolite.com/gitolite/}.
  \end{center}
\end{frame}


\begin{frame}{Credits}
  \begin{itemize}
  \item \textbf{Rike-Benjamin Schuppner}
  \item Zbigniew J\k{e}drzejewski-Szmek
  \item Tiziano Zito
  \item Bastian Venthur
  \item \url{http://progit.com}
  \item \url{apcmag.com}
  \item \url{lwn.net}
  \item \url{http://www.markus-gattol.name/ws/scm.html}
  \item \url{http://matthew-brett.github.io/pydagogue/gitwash/git_development.html}
  \end{itemize}
\end{frame}

\begin{frame}{I want to know more about \git!}
  Understanding how \git works:
  \begin{itemize}
  \item \git foundations, by Matthew Brett:
    \small
    \url{http://matthew-brett.github.com/pydagogue/foundation.html}
  \item  The \git parable, by Tom Preston-Werner:
    \small
  \url{http://tom.preston-werner.com/2009/05/19/the-git-parable.html}
  \end{itemize}
  Excellent guides:
  \begin{itemize}
  \item ``Pro Git'' book: \url{http://git-scm.com/book} (FREE)
  \item \git magic: \url{http://www-cs-students.stanford.edu/~blynn/gitmagic/}
  \end{itemize}
  Contributing to a project hosted on GitHub:
  \begin{itemize}
  \item ``Gitwash'', by Matthew Brett:
    \small
    \url{http://matthew-brett.github.io/pydagogue/gitwash/git_development.html}
  \end{itemize}
\end{frame}

\begin{frame}{Cool Stuff}
Gource:
  \begin{center}
    \url{http://code.google.com/p/gource/}
  \end{center}
\end{frame}

\begin{frame}{License}
  \begin{center}
    Copyright Emanuele Olivetti, 2014
  \end{center}
  \vspace{1em}
  \begin{center}
    This presentation is distributed under the license\\
    \textbf{Creative Commons \emph{Attribution} 3.0}\\
    \url{https://creativecommons.org/licenses/by/3.0/}
  \end{center}
  \vspace{1em}
  \begin{center}\small
    The diagrams of the branching example are taken from \emph{Pro
      Git}, (copyright S.Chacon, 2009) and are distributed under the
    license Creative Commons 3.0 Attribution-Non Commercial-Share
    Alike.
  \end{center}
\end{frame}

\end{document}
